%% Thanks to Bennet Goeckner for giving me his TeX template, which this is based on. 
%% These percent symbols tell the compiler to ignore the remainder of a given line.
%% We use them to write comments that will not appear in the finalized output.

%% The following tells the compiler which type of document we're making.
%% There are many options. ``Article'' is probably fine for our class.
\documentclass[12pt]{article}

%% After declaring the documentclass, we load some packages which give us 
%% some built-in commands and more functionality. 
%% The following is a list of packages that this file might use.
%% If a command you're using isn't working, try Googling it -- you might need to add a specific package.
%% I have included the standard ones that I like to load.
\usepackage[table]{xcolor}
\usepackage{amsmath}
\usepackage{amsthm}
\usepackage{amsfonts}
\usepackage{amssymb}
\usepackage{enumerate}
\usepackage{graphicx}
\usepackage{mdframed}
\usepackage{multicol}
\usepackage{verbatim}
\usepackage{tikz}
\usepackage[margin = .8in]{geometry}
\geometry{letterpaper}
\linespread{1.2}
\usepackage{cancel}

%% One of the nicest things about LaTeX is you can create custom macros. If  there is a long-ish expression that you will write often, it is nice to give it a shorter command.
%% For our common number systems.
\newcommand{\RR}{\mathbb{R}} %% The blackboard-bold R that you have seen used for real numbers is typeset by $\mathbb{R}$. This macro means that $\RR$ will yield the same result, and is much shorter to type.
\newcommand{\NN}{\mathbb{N}}
\newcommand{\ZZ}{\mathbb{Z}} 
\newcommand{\QQ}{\mathbb{Q}}

%% Your macros can even accept arguments. 
\newcommand\set[1]{\left\lbrace #1 \right\rbrace} %% In mathmode, if you write \set{STUFF}, then this will output {STUFF}, i.e. STUFF inside of a set
\newcommand\abs[1]{\left| #1 \right|} %% This will do the same but with vertical bars. I.e., \abs{STUFF} gives |STUFF|
\newcommand\parens[1]{\left( #1 \right)} %% Similar. \parens{STUFF} gives (STUFF)
\newcommand\brac[1]{\left[ #1 \right]} %% Similar. \brac{STUFF} gives [STUFF]
\newcommand\sol[1]{\begin{mdframed}
\emph{Solution.} #1
\end{mdframed}}
\newcommand\solproof[1]{\begin{mdframed}
\begin{proof} #1
\end{proof}
\end{mdframed}}

\newcommand\augmat[2]{\left[ \begin{array}{#1} #2 \end{array} \right]}
\newcommand\matopsinterchange[2]{#1 \longleftrightarrow #2}
\newcommand\matopsreplace[2]{#1 \xrightarrow{} #2}
\newcommand\matopsarrow[1]{\xrightarrow[\substack{#1}]{}}

\renewcommand\qedsymbol{$\blacksquare$}

\newcommand{\Pc}{\mathcal{P}}

\newcommand{\eps}{\varepsilon}

\newcommand\ceil[1]{\left\lceil #1 \right\rceil}
\newcommand\floor[1]{\left\lfloor #1 \right\rfloor}

\newcommand\defword[1]{\textit{\textbf{#1}}}

\graphicspath{ {./images/} }

%% A few more important commands:

%% You should start every proof with \begin{proof} and end it with \end{proof}.  
%%
%% Code inside single dollar signs will give in-line mathmode. I.e., $f(x) = x^2$ 
%% Code \[ \] will give mathmode centered on its own line.
%%
%% Other common commands:
%%	\begin{align*} and \end{align*} -- Good for multiline equations
%%	\begin{align} and \end{align} -- Same as above, but it will number the equations for easy reference
%%	\emph{italicized text here} and \textbf{bold text here} are also useful.
%%
%% Some very specific mathmode commands and their meanings:
%%	x \in A -- x is an element of A
%%	x \notin A -- x is not an element of A
%%	A \subseteq B -- A is a subset of B
%%	A \subsetneq B -- A is a proper subset of B
%%	x \equiv y \pmod{n} -- x is congruent to y mod n. 
%%	x \geq y and x \leq y -- Greater than or equal to and less than or equal to 
%%
%% You'll probably find lots of relevant commands in the question prompts. Also Google is your friend!

\begin{document}

\setlength{\parindent}{0pt}
\large
\textbf{Chapter 1: Real Numbers - Definitions}

\hrulefill

A \defword{function} $f$ from $A$ to $B$ is a subset of $A \times B$ such that for every $a \in A$ there is a unique $b \in B$ such that $(a, b) \in f$ (denoted $f : A \rightarrow B, f(a) = b$).

\hrulefill

Let $f : A \rightarrow B$ and $X \subseteq A$. The \defword{image} of $X$ under $f$ (denoted $f(X)$) is the set of all output values produced when $f$ is applied to each element of $A$.

If $f: A \rightarrow B$, and $X \subseteq A$, then $f(X) = \set{f(a) \in B \mid a \in X}$.

\hrulefill

Let $f : A \rightarrow B$ and $Y \subseteq B$. The \defword{preimage} of $Y$ under $f$ (denoted $f^{-1}(Y)$) is the set of all elements of $A$ that map to elements of $Y$.

If $f : A \rightarrow B$ and $Y \subseteq B$, then $f^{-1}(Y) = \set{a \in A \mid f(a) \in Y}$.

\hrulefill

Let $A \subseteq \RR$.

Then $b \in \RR$ is called an \defword{upper bound} for $A$ if $a \le b$ for every $a \in A$.

Then $b \in \RR$ is called a \defword{least upper bound} or \defword{supremum} for $A$
\\
(denoted $b = \sup A$) if:

\begin{enumerate}
\item $b$ is an upper bound for $A$, and
\item if $c$ is an upper bound for $A$, then $b \le c$.
\end{enumerate}

Similarly, a \defword{lower bound} and a \defword{greatest lower bound} or \defword{infimum}
\\
(denoted $\inf A$) are defined.

\hrulefill

Then $a_0 \in \RR$ is called a \defword{maximum} of $A$ if $a_0 \in A$ and $a \le a_0$ for every $a \in A$.

Similarly, a \defword{minimum} is defined.

\hrulefill

\defword{Equality of two real numbers}

Let $a, b \in \RR$. Then $a = b$ if and only if for every $\eps > 0, \abs{a - b} < \eps$.

\pagebreak

\textbf{Chapter 1: Real Numbers - Facts}

\hrulefill

\underline{Absolute value facts}: For $a, b \in \RR$:
\begin{enumerate}
\item $\abs{ab} = \abs{a} \cdot \abs{b}$
\item $\abs{a + b} \le \abs{a} + \abs{b}$ and $\abs{a - b} \le \abs{a} + \abs{b}$ (triangle inequality)
\item $\abs{\abs{a} - \abs{b}} \le \abs{a - b}$
\end{enumerate}

\hrulefill

\underline{Corollary}: Generalization of the triangle inequality

For $x_1, \dots, x_n, \abs{x_1 + \cdots + x_n} \le \abs{x_1} + \cdots + \abs{x_n}$.

\hrulefill

\underline{Fact}: Let $A \subseteq \RR$. If $A$ has a least upper bound, then it is unique.

\hrulefill

\underline{Lemma}: Let $a_0$ be an upper bound for set $A$. Then $a_0 = \sup A$ if and only if for every $\eps > 0$ there is $a \in A$ such that $a_0 - \eps < a$.

\hrulefill

\underline{Axiom of Completeness}: Every nonempty set of real numbers that has an upper bound has a least upper bound.

\hrulefill

\underline{Infimum Principle}: If $S$ is a nonempty subset of $\RR$ that has a lower bound, then $\inf S$ exists.

\hrulefill

$(\ast)$ \underline{Nested Interval Property}: Let $I_n = [a_n, b_n]$ and suppose for every $n \in \NN,
\\
I_{n + 1} \subseteq I_n$. Then $\bigcap_{n = 1}^{\infty} I_n \ne \emptyset$.

\hrulefill

$(\ast)$ \underline{Archimedean Property}:

\begin{enumerate}
\item For every $x \in \RR$ there is $n \in \NN$ such that $n > x$.
\\
$\forall_{x \in \RR} \exists_{n \in \NN} \text{ } n > x$

\item For every $y \in \RR^{+}$ there is $n \in \NN$ such that $\frac{1}{n} < y$.
\\
$\forall_{y \in \RR^{+}} \exists_{n \in \NN} \text{ } \frac{1}{n} < y$
\end{enumerate}

$(\ast)$ \underline{Density of $\QQ$ in $\RR$}:

For every two real numbers $a, b$ such that $a < b$, there is $r \in \QQ$ such that $a < r < b$.

\hrulefill

\underline{Corollary}: (Density of $\QQ \setminus \RR$ in $\RR$)

For every two real numbers $a, b$ such that $a < b$, there is $t \in \RR \setminus \QQ$ such that $a < t < b$.

\hrulefill

\underline{Fact}: There is no rational number $m$ such that $m^2 = 2$.

\hrulefill

\underline{Theorem}: There exists $\alpha \in \RR$ such that $\alpha^2 = 2$.

\hrulefill

\underline{Fact}: $\QQ$ is countable.

\hrulefill

\underline{Theorem}: $\RR$ is uncountable.

\pagebreak

\textbf{Chapter 2: Sequences - Definitions}

\hrulefill

A \defword{sequence} is a function whose domain is $\NN$ ($f : \NN \rightarrow \RR$).

\hrulefill

A sequence $(a_n)$ \defword{converges} to a real number $a$ (denoted $\lim a_n = a$) if

$\forall_{\eps > 0} \exists_{n_0 \in \NN} \forall_{n \ge n_0} \text{ } \abs{a_n - a} < \eps$.

\hrulefill

If a sequence does not converge for any real number, then the sequence \defword{diverges}.

\hrulefill

A sequence $(a_n)$ is \defword{bounded} if there exists $M > 0$ such that $\abs{a_n} \le M$ for every $n \in \NN$.

\hrulefill

A sequence $(a_n)$ is \defword{increasing} if $a_n \le a_{n + 1}$ for every $n \in \NN$.

A sequence $(a_n)$ is \defword{decreasing} if $a_n \ge a_{n + 1}$ for every $n \in \NN$.

\hrulefill

A sequence $(a_n)$ is \defword{bounded from above} if there exists $M \in \RR$ such that $a_n \le M$ for every $n \in \NN$.

A sequence $(a_n)$ is \defword{bounded from below} if there exists $M \in \RR$ such that $a_n \ge M$ for every $n \in \NN$.

\hrulefill

A sequence is \defword{monotone} if it is either increasing or decreasing.

\hrulefill

Let $(a_n)$ be a sequence of real numbers and let $n_1 < n_2 < \dots$ be an increasing (but not necessarily consecutive) sequence of natural numbers. Then
\\
$(a_{n_k}) = (a_{n_1}, a_{n_2}, \dots)$ is called a \defword{subsequence} of $(a_n)$.

\hrulefill

A sequence $(a_n)$ is called a \underline{Cauchy sequence} if for every $\eps > 0$ there is $n_0 \in \NN$ such that for all $m, n \ge n_0, \abs{a_n - a_m} < \eps$.

\pagebreak

Let $(b_n)$ be a sequence. An \defword{infinite series} is the expression $\Sigma_{n = 1}^{\infty} b_n = b_1 + b_2 + \cdots$.

The sequence of \defword{partial sums} $(s_m)$ is such that $s_m = b_1 + \cdots + b_m$.

We say that $\Sigma_{n = 1}^{\infty} b_n$ \defword{converges} to $B$ if $(s_m)$ converges to $B$.
\\
Then we write $\Sigma_{n = 1}^{\infty} b_n = B$.

\pagebreak

\textbf{Chapter 2: Sequences - Facts}

\hrulefill

\underline{Theorem}: If a limit exists, then it is unique.

\hrulefill

\underline{Theorem}: Every convergent sequence is bounded.

\hrulefill

\underline{Theorem}: Algebraic Limit Theorem

Let $\lim a_n = a$ and $\lim b_n = b$. Then:

\begin{enumerate}
\item $\lim (c \cdot a_n) = ca$, for every $c \in \RR$
\item $\lim (a_n + b_n) = a + b$
\item $\lim (a_n b_n) = ab$
\item $\lim \parens{\frac{a_n}{b_n}} = \frac{a}{b}$, provided $b \ne 0$
\end{enumerate}

\hrulefill

\underline{Theorem}: Order Limit Theorem

Let $\lim a_n = a$ and $\lim b_n = b$. Then:

\begin{enumerate}
\item If $a_n \ge 0$ for every $n$, then $a \ge 0$.
\item If $a_n \le b_n$ for every $n$, then $a \le b$.
\item If there exists $c$ such that $c \le b_n$, for all $n$, then $c \le b$.
\\
Similarly, if there exists $c$ such that $a_n \le c$, then $a \le c$.
\end{enumerate}

\hrulefill

\underline{Theorem}: Monotone Convergence Theorem

If a sequence is monotone and bounded, then it converges.

\hrulefill

$(\ast)$ \underline{Theorem}: MCT for increasing sequences

Let $(a_n)$ be an increasing sequence that is bounded from above. Then $(a_n)$ converges.

\pagebreak

\underline{Theorem}: Subsequences of a convergent sequence converge to the same limit as the original sequence.

\hrulefill

$(\ast)$ \underline{Theorem}: Bolzano-Weierstrass Theorem

Every bounded sequence contains a convergent subsequence.

\hrulefill

\underline{Lemma}: Every sequence contains a monotone subsequence.

\hrulefill

\underline{Theorem}: Every convergent sequence is a Cauchy sequence.

\hrulefill

\underline{Lemma}: Every Cauchy sequence is bounded.

\hrulefill

$(\ast)$ \underline{Theorem}: Cauchy Criterion

A sequence converges if and only if it is a Cauchy sequence.

\pagebreak

\textbf{Chapter 4: Continuity - Definitions}

\hrulefill

Let $x \in \RR$ and let $\eps > 0$. Then the \defword{$\eps$-neighborhood} is defined as:
\\
$V_{\eps}(x) = \set{y \mid \abs{x - y} < \eps}$.

\hrulefill

Let $A \subseteq \RR$. Then $x$ is called a \defword{limit point} of $A$ if for every $\eps > 0$, there exists $z \in A \cap V_{\eps}(x)$ such that $z \ne x$.

\hrulefill

Let $A$ be a non-degenerate interval or an interval with one point removed. Then the \defword{limit points of $A$} are points of $A$ together with the endpoints (if finite) and the removed point.

\hrulefill

Let $f : A \rightarrow \RR$ and let $c$ be a limit point of $A$. Then the \defword{functional limit} $lim_{x \rightarrow c} f(x) = L$ if for every $\eps > 0$, there is $\delta > 0$ such that if $0 < \abs{x - c} < \delta$ and $x \in A$, then $\abs{f(x) - L} < \eps$.

$\lim_{x \rightarrow c} f(x) = L \equiv \forall_{\eps > 0} \exists_{\delta > 0} \forall_{x \in A} 0 < \abs{x - c} < \delta \rightarrow \abs{f(x) - L} < \eps$

\hrulefill

A function $f : A \rightarrow \RR$ is \defword{continuous at a point} $c \in A$ if for every $\eps > 0$, there exists $\delta > 0$ such that if $\abs{x - c} < \delta$ and $x \in A$, then $\abs{f(x) - f(c)} < \eps$.

$\forall_{\eps > 0} \exists_{\delta > 0} \forall_{x \in A} \abs{x - c} < \delta \rightarrow \abs{f(x) - f(c)} < \eps$

\hrulefill

$f : A \rightarrow \RR$ is \defword{continuous on $A$} if it is continuous at $c$ for every $c \in A$.

\hrulefill

A function $f : A \rightarrow \RR$ is \defword{uniformly continuous} on $A$ if for every $\eps > 0$, there exists $\delta > 0$ such that for every $x, y \in A$, if $\abs{x - y} < \delta$, then $\abs{f(x) - f(y)} < \eps$.

$\forall_{\eps > 0} \exists_{\delta > 0} \forall_{x, y \in A} \abs{x - y} < \delta \rightarrow \abs{f(x) - f(y)} < \eps$

\pagebreak

\textbf{Chapter 4: Continuity - Facts}

\hrulefill

\underline{Theorem}: Let $A \subseteq \RR$. Then $x$ is a limit point of $A$ if and only if there exists a sequence $(a_n)$ that is contained in $A$ and $\lim a_n = x$ and $a_n \ne x$ (for every $n \in \NN$).

\hrulefill

\underline{Theorem}: Let $f : A \rightarrow \RR$ and let $c$ be a limit point of $A$. The following statements are equivalent:

\begin{enumerate}
\item $\lim_{x \rightarrow c} f(x) = L$

\item For every sequence $(x_n) \subseteq A$ satisfying $x_n \ne c$ and $\lim x_n = c$, we have $\lim f(x_n) = L$.
\end{enumerate}

\hrulefill

\underline{Corollary}: Let $f, g : A \rightarrow \RR$ and suppose $\lim_{x \rightarrow c} f(x) = L, \lim_{x \rightarrow c} g(x) = M$. Then:

\begin{enumerate}
\item $\lim_{x \rightarrow c} k \cdot f(x) = k \cdot L$ for every $k \in \RR$

\item $\lim_{x \rightarrow c} (f(x) + g(x)) = L + M$

\item $\lim_{x \rightarrow c} (f(x)g(x)) = L \cdot M$

\item $\lim_{x \rightarrow c} \frac{f(x)}{g(x)} = \frac{L}{M}$ provided $M \ne 0$
\end{enumerate}

\hrulefill

\underline{Corollary}: Divergence Criterion for Functional Limits

Let $f : A \rightarrow \RR$ and let $c$ be a limit point of $A$. If there exist two sequences $(x_n), (y_n)$ in $A$ such that $x_n \ne c, y_n \ne c, \lim x_n = \lim y_n = c$ but $\lim f(x_n) \ne \lim f(y_n)$, then $\lim_{x \rightarrow c} f(x)$ does not exist.

\pagebreak

\underline{Theorem}: Characterization of Continuity

Let $f : A \rightarrow \RR$ and let $c \in A$. The function is continuous at $c$ if and only if any of the following conditions is satisfied:

\begin{enumerate}
\item $\forall_{\eps > 0} \exists_{\delta > 0} (\abs{x - c} < \delta \land x \in A) \rightarrow \abs{f(x) - f(c)} < \eps$

\item For all $V_\eps(f(c))$, there exists $V_\delta(c)$ such that $x \in V_\delta(c) \cap A$ implies $f(x) \in V_{\eps}(f(c))$.

\item For all $(x_n) \rightarrow c$ with $x_n \in A, (f(x_n)) \rightarrow f(c)$.

In addition, if $c$ is a limit point of $A$, then the above are equivalent to:

\item $\lim_{x \rightarrow c} f(x) = f(c)$
\end{enumerate}

\hrulefill

\underline{Corollary}: Criterion for Discontinuity

Let $f : A \rightarrow \RR$ and let $c \in A$ (be a limit point of $A$). If there is a sequence $(x_n) \subseteq A$ such that $(x_n) \rightarrow c$ but $(f(x_n))$ does not converge to $f(c)$, then $f$ is not continuous at $c$.

\hrulefill

\underline{Theorem}: Algebraic Continuity Theorem

Let $f, g : A \rightarrow \RR$ be continuous at $c \in A$. Then:

\begin{enumerate}
\item $k \cdot f(x)$ is continuous at $c$ for every $k \in \RR$.

\item $f(x) + g(x)$ is continuous at $c$.

\item $f(x)g(x)$ is continuous at $c$.

\item $\frac{f(x)}{g(x)}$ is continuous at $c$ provided that the quotient is defined.
\end{enumerate}

\hrulefill

\underline{Theorem}: Let $f : A \rightarrow \RR$ and $g : B \rightarrow \RR$ be such that $f(A) \subseteq B$. If $f, g$ are continuous, then $g \circ f$ is a continuous function from $A$ to $\RR$.

\pagebreak

\underline{Theorem}: Bolzano-Weierstrass, version 2

Let $A$ be a closed bounded interval. Every sequence $(x_n) \subseteq A$ contains a convergent subsequence whose limit is in $A$.

$A = [a, b], z = \lim x_{n_k}, a \le z \le b$

\hrulefill

$(\ast)$ \underline{Theorem}: Extreme Value Theorem

Let $f : [a, b] \rightarrow \RR$ be continuous. Then there exist $\alpha, \beta \in [a, b]$ such that for all $x \in [a, b], f(\alpha) \le f(x) \le f(\beta)$.

\hrulefill

\underline{Theorem}: A function $f : A \rightarrow \RR$ fails to be uniformly continuous on $A$ if and only if there exists $\eps_0 > 0$ and two sequences $(x_n), (y_n)$ in $A$ satisfying $\abs{x_n - y_n} \rightarrow 0$ but $\abs{f(x_n) - f(y_n)} \ge \eps_0$.

\hrulefill

$(\ast)$ \underline {Theorem}: Let $f : [a, b] \rightarrow \RR$ be continuous. Then $f$ is uniformly continuous on $[a, b]$.

\hrulefill

$(\ast)$ \underline{Theorem}: Bolzano's Theorem

Let $f : [a, b] \rightarrow \RR$ be continuous. Suppose $f(a) < 0$ and $f(b) > 0$. Then there is $c \in (a, b)$ such that $f(c) = 0$.

\hrulefill

\underline{Theorem}: Intermediate Value Theorem

Let $f : [a, b] \rightarrow \RR$ be continuous. Let $L \in \RR$ be such that $\min \set{f(a), f(b)} < L < \max \set{f(a), f(b)}$. Then there is $c \in (a, b)$ such that $f(c) = L$.

\hrulefill

\underline{Theorem}: Let $f : [a, b] \rightarrow \RR$ be continuous. Then $f([a, b])$ (the range of $f$) is a closed interval.

\hrulefill

\underline{Continuity of the Inverse}: Let $f : [a, b] \rightarrow \RR$ be continuous and injective. Then $f([a, b]) = [c, d]$ and $f^{-1} : [c, d] \rightarrow [a, b]$.

\underline{Theorem}: With the setup as above, $f^{-1} : [c, d] \rightarrow [a, b]$ is continuous.

\pagebreak

\textbf{Chapter 5: The Derivative - Definitions}

\hrulefill

Let $g : A \rightarrow \RR$ be a function defined on interval $A$ and let $c \in A$. The \defword{derivative} of $g$ at $c$ is (provided that the limit exists):
\[
g'(c) = \lim \limits_{x \rightarrow c} \frac{g(x) - g(c)}{x - c}
\]

If $g$ exists for all points $c \in A$, then $g$ is \defword{differentiable} on $A$.

\hrulefill

\defword{Infinite limit.} $\lim_{x \rightarrow a} f(x) = \infty$ if for every $M > 0$, there exists $\delta > 0$ such that if $0 < \abs{x - a} < \delta$, then $f(x) > M$.

\pagebreak

\textbf{Chapter 5: The Derivative - Facts}

\hrulefill

\underline{Theorem}: If $g : A \rightarrow \RR$ is differentiable at $c$, then $g$ is continuous at $c$.

\hrulefill

\underline{Theorem}: Let $f, g : A \rightarrow \RR$ are differentiable at $c \in A$. Then:

\begin{enumerate}
\item $(f + g)'(c) = f'(c) + g'(c)$

\item For $k \in \RR, (k \cdot f)'(c) = k \cdot f'(c)$

\item $(fg)'(c) = f'(c)g(c) + f(c)g'(c)$ (Product Rule)

\item $\parens{\frac{f}{g}}'(c) = \frac{f'(c)g(c) - f(c)g'(c)}{(g(c))^2}$ provided $g(c) \ne 0$ (Quotient Rule)
\end{enumerate}

\hrulefill

\underline{Theorem}: The Chain Rule

Let $f : A \rightarrow B, g : B \rightarrow \RR$ satisfy $f(A) \subseteq B$. If $f$ is differentiable at $c \in A$ and $g$ is differentiable at $f(c) \in B$, then $g \circ f$ is differentiable at $c$ and
\[
(g \circ f)'(c) = g'(f(c)) f'(c)
\]

\hrulefill

\underline{Theorem}: Interior Extremum Theorem

Let $f$ be differentiable on $(a, b)$.

\begin{enumerate}
\item If $f$ attains a maximum value at some point $c \in (a, b)$, then $f'(c) = 0$.

\item If $f$ attains a minimum value at some point $c \in (a, b)$, then $f'(c) = 0$.
\end{enumerate}

\hrulefill

\underline{Theorem}: Darboux's Theorem

If $f$ is differentiable on an interval $[a, b]$ and if $\alpha$ satisfies $f'(a) < \alpha < f'(b)$
\\
(or $f'(b) < \alpha < f'(a)$), then there is a point $c \in (a, b)$ such that $f'(c) = \alpha$.

\hrulefill

\underline{Theorem}: Rolle's Theorem

Let $f : [a, b] \rightarrow \RR$ be continuous on $[a, b]$ and differentiable on $(a, b)$. If $f(a) = f(b)$, then there is a point $c \in (a, b)$ such that $f'(c) = 0$.

\pagebreak

\underline{Theorem}: Mean Value Theorem

If $f : [a, b] \rightarrow \RR$ is continuous on $[a, b]$ and differentiable on $(a, b)$, then there is $c \in (a, b)$ such that
\[
f'(c) = \frac{f(b) - f(a)}{b - a}.
\]

\hrulefill

\underline{Corollary}: Let $A$ be a non-degenerate interval ($A = [a, b], a < b$).
\\
If $g : A \rightarrow \RR$ is differentiable and satisfies $g'(x) = 0$ for all $x \in A$, then $g(x) = k$ for some $k \in \RR$.

\hrulefill

\underline{Corollary}: Let $A$ be a non-degenerate interval.
\\
If $f, g : A \rightarrow \RR$ are differentiable on $A$ and satisfy $f'(x) = g'(x)$ for every $x \in A$, then $f(x) = g(x) + k$ for some $k \in \RR$.

\hrulefill

\underline{Theorem}: Generalized Mean Value Theorem

If $f, g$ are continuous on $[a, b]$ and differentiable on $(a, b)$, then there exists $c \in (a, b)$ such that
\[
(f(b) - f(a)) g'(c) = (g(b) - g(a)) f'(c).
\]

In particular, if $g'(x) \ne 0$ for every $x \in (a, b)$, then
\[
\frac{f'(c)}{g'(c)} = \frac{f(b) - f(a)}{g(b) - g(a)}.
\]

\hrulefill

\underline{Theorem}: L'Hospital's Rule ($\frac{0}{0}$ case)

Let $f, g : [a, b] \rightarrow \RR$ be continuous on $[a, b]$ and differentiable on $(a, b)$.
\\
If $f(a) = g(a) = 0$ and $g'(x) \ne 0$ for all $x \ne a$, then
\[
\lim \limits_{x \rightarrow a} \frac{f'(x)}{g'(x)} = L \Rightarrow \lim \limits_{x \rightarrow a} \frac{f(x)}{g(x)} = L.
\]

\hrulefill

\underline{Theorem}: L'Hospital's Rule ($\frac{\infty}{\infty}$ case)

Let $f$ and $g$ be differentiable on $(a, b)$. Suppose $g'(x) \ne 0$ for all $x \in (a, b)$. If $\lim_{x \rightarrow a} g(x) = \infty$, then
\[
\lim \limits_{x \rightarrow a} \frac{f'(x)}{g'(x)} = L \Rightarrow \lim \limits_{x \rightarrow a} \frac{f(x)}{g(x)} = L.
\]

\pagebreak

\underline{Theorem}: Taylor's Theorem

Let $f : [a, b] \rightarrow \RR$. Suppose for some $n \in \NN, f^{(n)} (x)$ is continuous on $[a, b]$ and $f^{(n + 1)} (x)$ exists on $(a, b)$. Then for $x, x_0 \in [a, b]$, there is $c$ between $x, x_0$ such that

\[
f(x) = f(x_0) + f'(x_0)(x - x_0) + \cdots + \frac{f^{(n)} (x_0)}{n!} (x - x_0)^n + R_n (x)
\]
where
\[
R_n (x) = \frac{f^{(n + 1)} (c)}{(n + 1)!} (x - x_0)^{n + 1}.
\]

\pagebreak

\textbf{Chapter 6: Sequences of Functions - Definitions}

\hrulefill

\defword{Pointwise convergence.} For $n \in \NN$, let $f_n : A \rightarrow \RR$.
\\
The sequence $(f_n)$ \defword{converges pointwise} on $A$ to a function $f$ if for every
\\
$x \in A, (f_n (x))$ converges to $f(x)$.
\[
\forall_{x \in A} \forall_{\eps > 0} \exists_{n_0 \in \NN} \forall_{n \ge n_0} \abs{f_n (x) - f(x)} < \eps
\]

\hrulefill

\defword{Uniform convergence.} Let $(f_n)$ be a sequence of functions $f_n : A \rightarrow \RR$.
\\
Then $(f_n)$ \defword{converges uniformly} on $A$ to a function $f : A \rightarrow \RR$ if
\[
\forall_{\eps > 0} \exists_{n_0 \in \NN} \forall_{n \ge n_0} \forall_{x \in A} \abs{f_n (x) - f(x)} < \eps.
\]

$(f_n)$ does \underline{not} converge uniformly to $f$ on $A$ if
\[
\exists_{\eps_0 > 0} \forall_{n_0 \in \NN} \exists_{n \ge n_0} \exists_{x \in A} \abs{f_n (x) - f(x)} \ge \eps_0.
\]

\pagebreak

\textbf{Chapter 6: Sequences of Functions - Facts}

\hrulefill

\underline{Theorem}: Cauchy Criterion

A sequence of functions $(f_n), f_n : A \rightarrow \RR$ converges uniformly on $A$ if and only if
\[
\forall_{\eps > 0} \exists_{n_0 \in \NN} \forall_{n, m \ge n_0} \forall_{x \in A} \abs{f_n (x) - f_m (x)} < \eps.
\]

\hrulefill

\underline{Theorem}: Continuous Limit Theorem

Let $(f_n)$ be a sequence of functions $f_n : A \rightarrow \RR$ that converge uniformly on $A$ to $f$. If each $f_n$ is continuous at $c \in A$, then so is $f$.

\hrulefill

\underline{Theorem}: Differentiable Limit Theorem

Let $f_n \rightarrow f$ pointwise on $[a, b]$ and assume each $f_n$ is differentiable.
\\
If $(f_n ')$ converges uniformly on $[a, b]$ to $g$, then $f$ is differentiable and $f' = g$.

\pagebreak

\textbf{Chapter 7: Integration - Definitions}

\hrulefill

TODO

\pagebreak

\textbf{Chapter 7: Integration - Facts}

\hrulefill

TODO

\pagebreak

\textbf{Appendix A: Theorems to Memorize}

\hrulefill

\underline{Nested Interval Property (NIP)}

Let $I_n = [a_n, b_n]$ and suppose for every $n \in \NN, I_{n + 1} \subseteq {I_n}$. Then $\bigcap_{n = 1}^{\infty} I_n \ne \emptyset$.

\dotfill

High-Level Overview of the Proof:
\begin{enumerate}
\item Create a set containing the left endpoints of every interval, then use its properties to show that by the AoC, a sup exists.

\item Show that the sup is the element that all the intervals contain, making their intersection nonempty.
\end{enumerate}

\dotfill

\begin{proof}
$ $

Let $A = \set{a_n \mid n \in \NN}$.

Since $a_1 \in A, A \ne \emptyset$.

Since for every $n \in \NN, a_n \le b_1, b_1$ is an upper bound for $A$.

By the AoC, $\alpha = \sup A$ exists.

- - - - - - - - - -

Since $\alpha$ is an upper bound for $A$, for every $n \in \NN, a_n \le \alpha$.

Since $\alpha$ is the least upper bound for $A$ and for every $n \in \NN, b_n$ is an upper bound for $A, \alpha \le b_n$.

Thus, for every $n \in \NN, a_n \le \alpha \le b_n$, and so $\alpha \in I_n$ and $\alpha \in \bigcap_{n = 1}^{\infty} I_n$.

Therefore, $\bigcap_{n = 1}^{\infty} I_n \ne \emptyset$.
\end{proof}

\pagebreak

\underline{Monotone Convergence Theorem (MCT) for increasing sequences}

Let $(a_n)$ be an increasing sequence that is bounded from above. Then $(a_n)$ converges.

\dotfill

High-Level Overview of the Proof:

\begin{enumerate}
\item Create a set containing every element of the sequence, and use its properties to show that by the AoC, it has a sup.

\item Show that the limit of the sequence is that sup (starting with the sup lemma).
\end{enumerate}

\dotfill

\begin{proof}
$ $

Let $A = \set{a_n \mid n \in \NN}$.

Since $a_1 \in A, A \ne \emptyset$.

Since $(a_n)$ is bounded from above, $A$ has an upper bound.

By the AoC, $\alpha = \sup A$ exists.

- - - - - - - - - -

Let $\eps > 0$ be arbitrary.

Since $\alpha = \sup A$, there exists $n_0 \in \NN$ such that $\alpha - \eps < a_{n_0}$.

For $n \ge n_0$, since $a_{n_0} \le a_n$ and $a_n \le \alpha, -\eps < a_{n_0} - \alpha \le a_n - \alpha \le 0 < \eps$.

Thus, $\abs{a_n - \alpha} < \eps$, and so $(a_n)$ converges to $\alpha$.
\end{proof}

\pagebreak

\underline{Bolzano-Weierstrass Theorem (BW)}

Every bounded sequence has a convergent subsequence.

\dotfill

High-Level Overview of the Proof:

\begin{enumerate}
\item First prove the lemma that every sequence contains a monotone subsequence.
\begin{enumerate}
\item Define what a peak is, then show that for the case of a sequence having infinitely many peaks, it has a decreasing subsequence made up of all the peaks.

\item Show that for the case of a sequence having finitely many peaks, it has an increasing subsequence made up of all the points after the last peak (use induction to show that $a_{n_k} < a_{n_{k + 1}}$).
\end{enumerate}

\item Use the above lemma to show that a bounded sequence has a subsequence that is also bounded, and so by the MCT, that subsequence converges.
\end{enumerate}

\dotfill

\underline{Lemma}: Every sequence contains a monotone subsequence.
\begin{proof}
$ $

Let $(a_n)$ be a sequence.

Then $(a_m)$ is called a \underline{peak} if for every $l \ge m, a_l \le a_m$.

\begin{itemize}
\item \underline{Case 1}: There are infinitely many peaks in $(a_n)$.

Let $n_k$ be the index of the $k^{\text{th}}$ peak.

Since $n_k < n_{k + 1}, a_{n_k} \ge a_{n_{k + 1}}$.

Thus, $(a_{n_k})$ is decreasing.

- - - - - - - - - -

\item \underline{Case 2}: There are finitely many peaks in $(a_n)$.

Let $m$ be the index of the last peak ($m = 0$ if there are no peaks).

Let $n_1 = m + 1$.

Since $a_{n_1}$ is not a peak, there exists $n_2 > n_1$ such that $a_{n_1} < a_{n_2}$.

Suppose $a_{n_1}, \dots, a_{n_k}$ non-peaks have been chosen.

Since $a_{n_k}$ is not a peak, there exists $n_{k + 1} > n_k$ such that $a_{n_k} < a_{n_{k + 1}}$.

Thus, by induction, $(a_{n_k})$ is increasing.
\end{itemize}

Therefore, every sequence contains a monotone subsequence.
\end{proof}

- - - - - - - - - -

\underline{Proof of Bolzano-Weierstrass Theorem}

\begin{proof}
$ $

Let $(a_n)$ be a bounded sequence.

By the above lemma, $(a_n)$ contains a monotone subsequence $(a_{n_k})$.

Since $(a_n)$ is bounded, $(a_{n_k})$ is also bounded.

Therefore, by the MCT, $a_{n_k}$ converges.
\end{proof}

\pagebreak

\underline{Cauchy Criterion}

If a sequence is Cauchy, then it converges.

\dotfill

High-Level Overview of the Proof:

\begin{enumerate}
\item First prove the lemma that every Cauchy sequence is bounded.
\begin{enumerate}
\item Use the definition of the Cauchy sequence (with $\eps = 1$), and focus on the $n \ge n_0$ part to show that $\abs{a_n}$ is bounded by $1 + \abs{a_{n_0}}$.

\item Show that the entire sequence is bounded by the max of all the absolute values of sequence values before $a_{n_0}$ and the bound at and after $a_{n_0}$ ($1 + \abs{a_{n_0}}$).
\end {enumerate}

\item Use the lemma above and BW to show that a Cauchy sequence contains a convergent subsequence.

\item Show that the limit of that subsequence is the limit of the Cauchy sequence, meaning that the Cauchy sequence converges.
\begin{enumerate}
\item Use the definitions of convergence for the subsequence and a Cauchy sequence with $\frac{\eps}{2}$ for the $\eps$-values to setup inequalities to be used later.

\item Choose an index to be the further out of the subsequence and Cauchy sequence convergence points ($k_0$ and $n_0$), and transform the inequalities to use that index.

\item Show that the Cauchy sequence converges to the value to which the subsequence converges, using the definition of convergence with the inequalities.
\end{enumerate}
\end{enumerate}

\dotfill

\underline{Lemma}: Every Cauchy sequence is bounded.

\begin{proof}
$ $

Let $(a_n)$ be a Cauchy sequence.

Then there exists $n_0 \in \NN$ such that for every $n, m \ge n_0, \abs{a_n - a_m} < 1$.

In particular, for every $n \ge n_0, \abs{a_n - a_{n_0}} < 1$.

Then for every $n \ge n_0, \abs{a_n} = \abs{a_n - a_{n_0} + a_{n_0}} \le \abs{a_n - a_{n_0}} + \abs{a_{n_0}} < 1 + \abs{a_{n_0}}$.

- - - - - - - - - -

Let $M = \max \set{\abs{a_1}, \abs{a_2}, \dots, \abs{a_{n_0 - 1}}, 1 + \abs{a_{n_0}}}$.

Then for every $n \in \NN, \abs{a_n} \le M$.

Therefore, $(a_n)$ is bounded.
\end{proof}

- - - - - - - - - -

\underline{Proof of the Cauchy Criterion}

\begin{proof}
$ $

Let $(a_n)$ be a Cauchy sequence.

By the lemma above, $(a_n)$ is bounded.

By Bolzano-Weierstrass, $(a_n)$ contains a convergent subsequence $(a_{n_k})$.

- - - - - - - - - -

Let $a = \lim a_{n_k}$.

Let $\eps > 0$ be arbitrary.

Since $\lim a_{n_k} = a$, there exists $k_0 \in \NN$ such that for every $k \ge k_0, \abs{a_{n_k} - a} < \frac{\eps}{2}$.

Since $(a_n)$ is a Cauchy sequence, there exists $n_0 \in \NN$ such that for every
\\
$n, m \ge n_0, \abs{a_n - a_m} < \frac{\eps}{2}$.

- - - - - - - - - -

Let $l = \max \set{k_0, n_0}$.

Then for $n \ge n_0, \abs{a_{n_l} - a} < \frac{\eps}{2}$ and $\abs{a_n - a_{n_l}} < \frac{\eps}{2}$.

- - - - - - - - - -

Then for $n \ge n_0, \abs{a_n - a} = \abs{a_n - a_{n_l} + a_{n_l} - a} \le \abs{a_n - a_{n_l}} + \abs{a_{n_l} - a}
\\
< \frac{\eps}{2} + \frac{\eps}{2} = \eps$.

Therefore, $\lim a_n = a$, and so $(a_n)$ converges.
\end{proof}

\pagebreak

\underline{Extreme Value Theorem}

Let $f : [a, b] \rightarrow \RR$ be continuous. Then there exist $\alpha, \beta \in [a, b]$ such that for all $x \in [a, b], f(\alpha) \le f(x) \le f(\beta)$.

\dotfill

High-Level Overview of the Proof:

\begin{enumerate}
\item First show that $f$ is bounded on $[a, b]$.
\begin{enumerate}
\item By contradiction, we suppose $f$ is unbounded on the interval by constructing a sequence in $[a, b]$ whose absolute valued function output values are at or above their respective index values, which is possible because this assumption allows the function values to grow without bound on the closed interval.

\item Use BW version 2 to show that a subsequence of that sequence converges to a value in $[a, b]$, and since $f$ is continuous, the subsequence's function output values also converge.

\item This contradicts the assumption that the $f$ was unbounded, as the absolute valued function output values of the subsequence also grow without bound under that original assumption, and thus shouldn't converge.
\end{enumerate}

\item By the AoC, the set of function output values on $[a, b]$ has a sup that we'll show is the max of the function output values, and we'll find the input value of $f$ associated with that max value.

\item Construct a sequence in $[a, b]$ using the sup lemma with the function output values of that sequence (with $\eps = \frac{1}{n}$), and bound that statement from above with the sup.

\item Use BW version 2 to show that the sequence has a subsequence that converges to the input value, and converting the above statement to use the subsequence and applying the squeeze theorem shows that the function output values of the subsequence converge to the sup.

\item Show that since $f$ is continuous, the function output values of the subsequence also converge to the function value at the input value, and so the function values on $[a, b]$ have the sup as an upper bound, which is the same as the function value at the input value.
\end{enumerate}

\dotfill

\begin{proof}
$ $

\begin{enumerate}
\item We will show that $f$ is bounded on $[a, b]$.

By contradiction.

Suppose for every $n \in \NN$, there exists $x_n \in [a, b]$ such that $\abs{f(x_n)} \ge n$. $(\ast)$

- - - - - - - - - -

By BW version 2, $(x_n)$ contains a convergent subsequence $(x_{n_k})$ such that $x = \lim x_{n_k} \in [a, b]$.

Since $f$ is continuous on $[a, b], \lim f(x_{n_k}) = f(x)$.

- - - - - - - - - -

This contradicts $(\ast)$, as by $(\ast), \abs{f(x_{n_k})} \ge n_k \ge k$.

- - - - - - - - - -

\item By the AoC, $M = \sup \set{f(x) \mid x \in [a, b]}$ exists.

- - - - - - - - - -

Then for every $n \in \NN$, there exists $x_n \in [a, b]$ such that $M - \frac{1}{n} < f(x_n)$.

Thus, $M - \frac{1}{n} < f(x_n) \le M$.

- - - - - - - - - -

By BW version 2, $(x_n)$ contains a convergent subsequence $(x_{n_k})$ such that $\beta = \lim x_{n_k} \in [a, b]$.

Then $M - \frac{1}{k} \le M - \frac{1}{n_k} \le f(x_{n_k}) \le M$.

Thus, by the squeeze theorem, $\lim f(x_{n_k}) = M$.

- - - - - - - - - -

At the same time, since $f$ is continuous on $[a, b], \lim f(x_{n_k}) = f(\beta)$.

Thus, there exists $\beta \in [a, b]$ such that $f(\beta) = M$.

Therefore, for every $x \in [a, b], f(x) \le M = f(\beta)$.
\end{enumerate}
\end{proof}

\pagebreak

\underline{Bolzano's Theorem}

Let $f: [a, b] \rightarrow \RR$ be continuous. Suppose $f(a) < 0$ and $f(b) > 0$. Then there is $c \in (a, b)$ such that $f(c) = 0$.

\dotfill

High-Level Overview of the Proof:

\begin{enumerate}
\item Use induction to create a sequence of closed intervals, each of which uses the previous interval's midpoint to be one of the new interval's endpoints. This is chosen as such to keep each interval's left endpoint negative and its right endpoint non-negative.

\item Use the NIP to show that both the left and right endpoints of the intervals converge to a value.

\item Use the definition of continuity and the order of limits theorem to show that the function output value at the above determined value is zero.
\end{enumerate}

\dotfill

\begin{proof}
$ $

Let $a_1 = a, b_1 = b$, and $I_n = [a_1, b_1]$.

Suppose a sequence of closed intervals $I_1, \dots, I_n$ with $I_j = [a_j, b_j]$ have been constructed such that $I_1 \supseteq I_2 \supseteq \dots \supseteq I_n, f(a_j) < 0$, and $f(b_j) > 0$.

Let $z = \frac{a_n + b_n}{2}$.

\begin{enumerate}
\item If $f(z) < 0$, then $a_{n + 1} = z$ and $b_{n + 1} = b_n$.

\item If $f(z) \ge 0$, then $a_{n + 1} = a_n$ and $b_{n + 1} = z$.
\end{enumerate}

Let $I_{n + 1} = [a_{n + 1}, b_{n + 1}]$.

Then $I_{n + 1}$ is a closed interval, $I_{n + 1} \subseteq I_n, f(a_{n + 1}) < 0$, and $f(b_{n + 1}) \ge 0$.

Thus, by induction we have a nested sequence of closed intervals $I_1 \supseteq I_2 \supseteq \dots$.

- - - - - - - - - -

By the NIP, $\bigcap_{n = 1}^{\infty} I_n \ne \emptyset$.

Let $c \in \bigcap_{n = 1}^{\infty} I_n$.

Then for every $n \in \NN, a_n \le c \le b_n$.

Since $\lim \text{length}(I_n) = 0, \lim a_n = c = \lim b_n$.

- - - - - - - - - -

Since $f$ is continuous on $[a, b], \lim f(a_n) = f(c) = f(b_n)$.

Since $f(a_n) < 0$ for every $n \in \NN$, by the order of limits theorem, $f(c) \le 0$.

Since $f(b_n) \ge 0$ for every $n \in \NN, f(c) \ge 0$.

Therefore, $f(c) = 0$ and $c \in (a, b)$.
\end{proof}

\pagebreak

\underline{The Chain Rule}

Let $f : A \rightarrow B, g : B \rightarrow \RR$ satisfy $f(A) \subseteq B$. If $f$ is differentiable at $c \in A$ and $g$ is differentiable at $f(c) \in B$, then $g \circ f$ is differentiable at $c$ and
\[
(g \circ f)'(c) = g'(f(c)) f'(c)
\]

\dotfill

High-Level Overview of the Proof:

TODO

\dotfill

\begin{proof}
$ $

Let $d : B \rightarrow \RR$ be defined as follows:
\[
d(y) =
\begin{cases}
	\frac{g(y) - g(f(c))}{y - f(c)} & \text{if }y \ne f(c) \\
	g'(f(c)) & \text{if }y = f(c)
\end{cases}
\]

Then $\lim_{y \rightarrow f(c)} d(y) = g'(f(c))$, so $d$ is continuous at $f(c)$.

Then we have $(\ast)$ $d(y) (y - f(c)) = g(y) - g(f(c))$ holds for every $y \in B$ (including $y = f(c)$).

Then for $t \in A$ with $y = f(t)$, $(\ast)$ becomes
\\
$d(f(t)) (f(t) - f(c)) = g(f(t)) - g(f(c))$.

Thus, if $t \ne c, \frac{g(f(t)) - g(f(c))}{t - c} = d(f(t)) \cdot \frac{f(t) - f(c)}{t - c}$.

Since $f$ is differentiable at $c$, it is continuous at $c$, and $d(y)$ is continuous at $f(c)$.

Thus, $(d \circ f)$ is continuous at $c$.

As a result, $\lim_{t \rightarrow c} d(f(t)) = d(f(c)) = g'(f(c))$.

Therefore, $(g \circ f)'(c) = \lim_{t \rightarrow c} \frac{g(f(t)) - g(f(c))}{t - c} = g'(f(c)) f'(c)$.
\end{proof}

\pagebreak

\underline{Rolle's Theorem}

Let $f : [a, b] \rightarrow \RR$ be continuous on $[a, b]$ and differentiable on $(a, b)$. If $f(a) = f(b)$, then there exists $c \in (a, b)$ such that $f'(c) = 0$.

\dotfill

High-Level Overview of the Proof:

TODO

\dotfill

\begin{proof}
$ $

Since $f : [a, b] \rightarrow \RR$ is continuous, by the EVT, the max and min of $f$ are attained on $[a, b]$.

\begin{enumerate}
\item
If both the max and min of $f$ are at $a$ and $b$, then
\\
$\text{max of }f = f(a) = f(b) = \text{min of }f$.

Then $f$ is necessarily constant.

Thus, $f'(c) = 0$ for every $c \in (a, b)$.

\item
Otherwise, by the IET, there exists $c \in (a, b)$ such that $f'(c) = 0$.
\end{enumerate}

Therefore, there exists $c \in (a, b)$ such that $f'(c) = 0$.
\end{proof}

\pagebreak

\underline{Mean Value Theorem}

If $f : [a, b] \rightarrow \RR$ is continuous on $[a, b]$ and differentiable on $(a, b)$, then there exists $c \in (a, b)$ such that
\[
f'(c) = \frac{f(b) - f(a)}{b - a}
\]

\dotfill

High-Level Overview of the Proof:

TODO

\dotfill

\begin{proof}
$ $

Let $d(x) = f(x) - \parens{\frac{f(b) - f(a)}{b - a} (x - a) + f(a)}$.

Then $d$ is continuous on $[a, b]$ and differentiable on $(a, b)$.

In addition, $d(a) = 0 = d(b)$, and $d'(x) = f'(x) - \frac{f(b) - f(a)}{b - a}$.

By Rolle's Theorem, there exists $c \in (a, b)$ such that $d'(c) = 0$, and so
\\
$0 = d'(c) = f'(c) - \frac{f(b) - f(a)}{b - a}$.

Therefore, $f'(c) = \frac{f(b) - f(a)}{b - a}$.
\end{proof}

\pagebreak

\underline{Continuous Limit Theorem}

Let $(f_n)$ be a sequence of functions $f_n : A \rightarrow \RR$ that converge uniformly
\\
on $A$ to $f$. If each $f_n$ is continuous at $c \in A$, then so is $f$.

\dotfill

High-Level Overview of the Proof:

TODO

\dotfill

\begin{proof}
$ $

Let $\eps > 0$.

Since $(f_n)$ converges uniformly on $A$ to $f$, there exists $n_0 \in \NN$ such that for every $n \ge n_0$ and $x \in A, \abs{f_n(x) - f(x)} < \frac{\eps}{3}$.

Since $f_{n_0}$ is continuous at $c$, there exists $\delta > 0$ such that if $\abs{x - c} < \delta$, then $\abs{f_{n_0}(x) - f_{n_0}(c)} < \frac{\eps}{3}$.

Then if $\abs{x - c} < \delta$, then
\\
$\abs{f(x) - f(c)} \le \abs{f(x) - f_{n_0}(x)} + \abs{f_{n_0}(x) - f_{n_0}(c)} + \abs{f_{n_0}(c) - f(c)}
\\
< \frac{\eps}{3} + \frac{\eps}{3} + \frac{\eps}{3} = \eps$.

Therefore, $f$ is continuous at $c$.
\end{proof}

\pagebreak

\underline{Integrable Limit Theorem}

Assume $f_n \rightarrow f$ uniformly on $[a, b]$ and each $f_n$ is integrable. Then $f$ is integrable and $\lim \int_{a}^{b} f_n = \int_{a}^{b} f$.

\dotfill

High-Level Overview of the Proof:

TODO

\dotfill

\begin{proof}
$ $

Let $\eps > 0$ be arbitrary.

Since $(f_n)$ converges uniformly to $f$ on $[a, b]$, there exists $n_0 \in \NN$ such that for every $n \ge n_0$ and $x \in [a, b], \abs{f_n(x) - f(x)} < \frac{\eps}{3(b - a)}$.

Then $-\frac{\eps}{3(b - a)} < f(x) - f_n(x) < \frac{\eps}{3(b - a)}$.

$f_{n_0}$ is integrable on $[a, b]$, and so there exists a partition $P_0$ of $[a, b]$ such that $U(f_{n_0}, P_0) - L(f_{n_0}, P_0) < \frac{\eps}{3}$.

We have $U(f + g, P) \le U(f, P) + U(g, P)$ and $L(f + g, P) \ge L(f, P) + L(g, P)$.

Then $U(f, P_0) - L(f, P_0) = U(f - f_{n_0} + f_{n_0}, P_0) - L(f - f_{n_0} + f_{n_0}, P_0)
\\
\le U(f - f_{n_0}, P_0) + (U(f_{n_0}, P_0) - L(f_{n_0}, P_0)) - L(f - f_{n_0}, P_0)$.

We have $U(f - f_{n_0}, P_0) \le \frac{\eps}{3(b - a)} (b - a) = \frac{\eps}{3}$ and $L(f - f_{n_0}, P_0) \ge -\frac{\eps}{3(b - a)} (b - a) = -\frac{\eps}{3}$.

Therefore, $U(f, P_0) - L(f, P_0) < \frac{\eps}{3} + \frac{\eps}{3} + \frac{\eps}{3} = \eps$.

For $n \ge n_0, \abs{\int_a^b f_n - \int_a^b f} = \abs{\int_a^b (f_n - f)} \le \int_a^b \abs{f_n - f} \le \frac{\eps}{3(b - a)} (b - a) < \eps$.
\end{proof}

\pagebreak

\underline{The Fundamental Theorem of Calculus (Part 1)}

If $f : [a, b] \rightarrow \RR$ is integrable and $F : [a, b] \rightarrow \RR$ satisfies $F'(x) = f(x)$ for every $x \in [a, b]$, then $\int_a^b f(x) dx = F(b) - F(a)$.

\dotfill

High-Level Overview of the Proof:

TODO

\dotfill

\begin{proof}
$ $

Since $F$ is differentiable on $[a, b]$, it is also continuous on $[a, b]$.

Let $P = \set{x_0, x_1, \dots, x_n}$ be a partition of $[a, b]$.

For every $k = 1, \dots, n$, by the MVT, there exists $t_k \in [x_{k - 1}, x_k]$ such that
\\
$F'(t_k) = \frac{F(x_k) - F(x_{k - 1})}{x_k - x_{k - 1}}$.

Then $f(t_k) = \frac{F(x_k) - F(x_{k - 1})}{x_k - x_{k - 1}}$, and so $F(x_k) - F(x_{k - 1}) = f(t_k) \Delta x_k$.

Then $m_k \le f(t_k) \le M_k$.

Therefore, $\sum_{k = 1}^n (F(x_k) - F(x_{k - 1})) = \sum_{k = 1}^n f(t_k) \Delta x_k \le \sum_{k = 1}^n M_k \Delta x_k = U(f, P)$ and $\sum_{k = 1}^n (F(x_k) - F(x_{k - 1})) \ge \sum_{k = 1}^n m_k \Delta x_k = L(f, P)$.

However, $\sum_{k = 1}^n (F(x_k) - F(x_{k - 1})) = F(x_n) - F(x_0) = F(b) - F(a)$.

Thus, $L(f, P) \le F(b) - F(a) \le U(f, P)$, and so $L(f) \le F(b) - F(a) \le U(f)$.

Since $f$ is integrable on $[a, b], L(f) = U(f)$, and so $F(b) - F(a) = L(f) = \int_a^b f$.
\end{proof}

\pagebreak

\textbf{Appendix B: Theorems no longer needed for the final}

\hrulefill

\underline{Archimedean Property (ArP)}

For every $x \in \RR$, there exists $n \in \NN$ such that $n > x$.

\dotfill

High-Level Overview of the Proof:

\begin{enumerate}
\item By contradiction, assume the negation of the statement and argue that $x$ is an upper bound for $\NN$, and so by the AoC, a sup exists for $\NN$.

\item Show that there exists an element in $\NN$ that is greater than the sup (using the sup lemma with $\eps = 1$), which is a contradiction.
\end{enumerate}

\dotfill

\begin{proof}
$ $

By contradiction.

Suppose there exists $x \in \RR$ such that for every $n \in \NN, n \le x$.

Then $x$ is an upper bound for $\NN$.

Clearly, $\NN \ne \emptyset$.

By the AoC, $\alpha = \sup \NN$ exists.

- - - - - - - - - -

Then there exists $n \in \NN$ such that $\alpha - 1 < n$.

Then $\alpha < n + 1$.

Since $n \in \NN, n + 1 \in \NN$.

This contradicts the fact that $\alpha$ is an upper bound for $\NN$.
\end{proof}

\pagebreak

\underline{Theorem on continuous functions on a closed interval being uniformly continuous}

Let $f : [a, b] \rightarrow \RR$ be continuous. Then $f$ is uniformly continuous on $[a, b]$.

\dotfill

High-Level Overview of the Proof:

\begin{enumerate}
\item By contradiction, suppose $f$ is not uniformly continuous on $[a, b]$, and use the theorem for a function failing to be uniformly continuous.

\item Use BW version 2 and the triangle inequality to show that a subsequence of one of the above sequences converges to the same value as a subsequence of the other.

\item Use the definition of continuity and the triangle inequality to show that the absolute value of the difference of function output values between the two subsequences converges to zero, contradicting the assumption that they should converge to a positive real number.
\end{enumerate}

\dotfill

\begin{proof}
$ $

By contradiction.

Suppose there exist $\eps_0 > 0$ and $(x_n), (y_n)$ contained in $[a, b]$ such that
\\
$\lim \abs{x_n - y_n} = 0$, but $\abs{f(x_n) - f(y_n)} \ge \eps_0$ for every $n \in \NN$. $(\ast)$

- - - - - - - - - -

By BW version 2, there exists a subsequence $(x_{n_k})$ of $(x_n)$ that converges and such that $x = \lim x_{n_k} \in [a, b]$.

We have $\abs{y_{n_k} - x} \le \abs{y_{n_k} - x_{n_k}} + \abs{x_{n_k} - x} \rightarrow 0$.

Thus, $\lim y_{n_k} = x$.

- - - - - - - - - -

Since $f$ is continuous on $[a, b], \lim f(x_{n_k}) = f(x) = \lim f(y_{n_k})$, and so
\\
$\abs{f(x_{n_k}) - f(y_{n_k})} \le \abs{f(x_{n_k}) - f(x)} + \abs{f(y_{n_k}) - f(x)} \rightarrow 0$, contradicting $(\ast)$.
\end{proof}

\end{document}
