%% Thanks to Bennet Goeckner for giving me his TeX template, which this is based on. 
%% These percent symbols tell the compiler to ignore the remainder of a given line.
%% We use them to write comments that will not appear in the finalized output.

%% The following tells the compiler which type of document we're making.
%% There are many options. ``Article'' is probably fine for our class.
\documentclass[12pt]{article}

%% After declaring the documentclass, we load some packages which give us 
%% some built-in commands and more functionality. 
%% The following is a list of packages that this file might use.
%% If a command you're using isn't working, try Googling it -- you might need to add a specific package.
%% I have included the standard ones that I like to load.
\usepackage[table]{xcolor}
\usepackage{amsmath}
\usepackage{amsthm}
\usepackage{amsfonts}
\usepackage{amssymb}
\usepackage{enumerate}
\usepackage{graphicx}
\usepackage{mdframed}
\usepackage{multicol}
\usepackage{verbatim}
\usepackage{tikz}
\usepackage[margin = .8in]{geometry}
\geometry{letterpaper}
\linespread{1.2}
\usepackage{cancel}

%% One of the nicest things about LaTeX is you can create custom macros. If  there is a long-ish expression that you will write often, it is nice to give it a shorter command.
%% For our common number systems.
\newcommand{\RR}{\mathbb{R}} %% The blackboard-bold R that you have seen used for real numbers is typeset by $\mathbb{R}$. This macro means that $\RR$ will yield the same result, and is much shorter to type.
\newcommand{\NN}{\mathbb{N}}
\newcommand{\ZZ}{\mathbb{Z}} 
\newcommand{\QQ}{\mathbb{Q}}

%% Your macros can even accept arguments. 
\newcommand\set[1]{\left\lbrace #1 \right\rbrace} %% In mathmode, if you write \set{STUFF}, then this will output {STUFF}, i.e. STUFF inside of a set
\newcommand\abs[1]{\left| #1 \right|} %% This will do the same but with vertical bars. I.e., \abs{STUFF} gives |STUFF|
\newcommand\parens[1]{\left( #1 \right)} %% Similar. \parens{STUFF} gives (STUFF)
\newcommand\brac[1]{\left[ #1 \right]} %% Similar. \brac{STUFF} gives [STUFF]
\newcommand\sol[1]{\begin{mdframed}
\emph{Solution.} #1
\end{mdframed}}
\newcommand\solproof[1]{\begin{mdframed}
\begin{proof} #1
\end{proof}
\end{mdframed}}

\newcommand\augmat[2]{\left[ \begin{array}{#1} #2 \end{array} \right]}
\newcommand\matopsinterchange[2]{#1 \longleftrightarrow #2}
\newcommand\matopsreplace[2]{#1 \xrightarrow{} #2}
\newcommand\matopsarrow[1]{\xrightarrow[\substack{#1}]{}}

\renewcommand\qedsymbol{$\blacksquare$}

\newcommand{\Pc}{\mathcal{P}}

\newcommand{\eps}{\varepsilon}

\newcommand\ceil[1]{\left\lceil #1 \right\rceil}
\newcommand\floor[1]{\left\lfloor #1 \right\rfloor}

\newcommand\defword[1]{\textit{\textbf{#1}}}

\graphicspath{ {./images/} }

%% A few more important commands:

%% You should start every proof with \begin{proof} and end it with \end{proof}.  
%%
%% Code inside single dollar signs will give in-line mathmode. I.e., $f(x) = x^2$ 
%% Code \[ \] will give mathmode centered on its own line.
%%
%% Other common commands:
%%	\begin{align*} and \end{align*} -- Good for multiline equations
%%	\begin{align} and \end{align} -- Same as above, but it will number the equations for easy reference
%%	\emph{italicized text here} and \textbf{bold text here} are also useful.
%%
%% Some very specific mathmode commands and their meanings:
%%	x \in A -- x is an element of A
%%	x \notin A -- x is not an element of A
%%	A \subseteq B -- A is a subset of B
%%	A \subsetneq B -- A is a proper subset of B
%%	x \equiv y \pmod{n} -- x is congruent to y mod n. 
%%	x \geq y and x \leq y -- Greater than or equal to and less than or equal to 
%%
%% You'll probably find lots of relevant commands in the question prompts. Also Google is your friend!

\begin{document}

\setlength{\parindent}{0pt}
\large
\textbf{Chapter 1: Real Numbers - Definitions}

\hrulefill

A \defword{function} $f$ from $A$ to $B$ is a subset of $A \times B$ such that for every $a \in A$ there is a unique $b \in B$ such that $(a, b) \in f$ (denoted $f : A \rightarrow B, f(a) = b$).

\hrulefill

Let $f : A \rightarrow B$ and $X \subseteq A$. The \defword{image} of $X$ under $f$ (denoted $f(X)$) is the set of all output values produced when $f$ is applied to each element of $A$.

If $f: A \rightarrow B$, and $X \subseteq A$, then $f(X) = \set{f(a) \in B \mid a \in X}$.

\hrulefill

Let $f : A \rightarrow B$ and $Y \subseteq B$. The \defword{preimage} of $Y$ under $f$ (denoted $f^{-1}(Y)$) is the set of all elements of $A$ that map to elements of $Y$.

If $f : A \rightarrow B$ and $Y \subseteq B$, then $f^{-1}(Y) = \set{a \in A \mid f(a) \in Y}$.

\hrulefill

Let $A \subseteq \RR$.

Then $b \in \RR$ is called an \defword{upper bound} for $A$ if $a \le b$ for every $a \in A$.

Then $b \in \RR$ is called a \defword{least upper bound} or \defword{supremum} for $A$
\\
(denoted $b = \sup A$) if:

\begin{enumerate}
\item $b$ is an upper bound for $A$, and
\item if $c$ is an upper bound for $A$, then $b \le c$.
\end{enumerate}

Similarly, a \defword{lower bound} and a \defword{greatest lower bound} or \defword{infimum}
\\
(denoted $\inf A$) are defined.

\hrulefill

Then $a_0 \in \RR$ is called a \defword{maximum} of $A$ if $a_0 \in A$ and $a \le a_0$ for every $a \in A$.

Similarly, a \defword{minimum} is defined.

\hrulefill

\defword{Equality of two real numbers}

Let $a, b \in \RR$. Then $a = b$ if and only if for every $\eps > 0, \abs{a - b} < \eps$.

\pagebreak

\textbf{Chapter 1: Real Numbers - Facts}

\hrulefill

\underline{Absolute value facts}: For $a, b \in \RR$:
\begin{enumerate}
\item $\abs{ab} = \abs{a} \cdot \abs{b}$
\item $\abs{a + b} \le \abs{a} + \abs{b}$ and $\abs{a - b} \le \abs{a} + \abs{b}$ (triangle inequality)
\item $\abs{\abs{a} - \abs{b}} \le \abs{a - b}$
\end{enumerate}

\hrulefill

\underline{Corollary}: Generalization of the triangle inequality

For $x_1, \dots, x_n, \abs{x_1 + \cdots + x_n} \le \abs{x_1} + \cdots + \abs{x_n}$.

\hrulefill

\underline{Fact}: Let $A \subseteq \RR$. If $A$ has a least upper bound, then it is unique.

\hrulefill

\underline{Lemma}: Let $a_0$ be an upper bound for set $A$. Then $a_0 = \sup A$ if and only if for every $\eps > 0$ there is $a \in A$ such that $a_0 - \eps < a$.

\hrulefill

\underline{Axiom of Completeness}: Every nonempty set of real numbers that has an upper bound has a least upper bound.

\hrulefill

\underline{Infimum Principle}: If $S$ is a nonempty subset of $\RR$ that has a lower bound, then $\inf S$ exists.

\hrulefill

$(\ast)$ \underline{Nested Interval Property}: Let $I_n = [a_n, b_n]$ and suppose for every $n \in \NN,
\\
I_{n + 1} \subseteq I_n$. Then $\bigcap_{n = 1}^{\infty} I_n \ne \emptyset$.

\hrulefill

$(\ast)$ \underline{Archimedean Property}:

\begin{enumerate}
\item For every $x \in \RR$ there is $n \in \NN$ such that $n > x$.
\\
$\forall_{x \in \RR} \exists_{n \in \NN} \text{ } n > x$

\item For every $y \in \RR^{+}$ there is $n \in \NN$ such that $\frac{1}{n} < y$.
\\
$\forall_{y \in \RR^{+}} \exists_{n \in \NN} \text{ } \frac{1}{n} < y$
\end{enumerate}

$(\ast)$ \underline{Density of $\QQ$ in $\RR$}:

For every two real numbers $a, b$ such that $a < b$, there is $r \in \QQ$ such that $a < r < b$.

\hrulefill

\underline{Corollary}: (Density of $\QQ \setminus \RR$ in $\RR$)

For every two real numbers $a, b$ such that $a < b$, there is $t \in \RR \setminus \QQ$ such that $a < t < b$.

\hrulefill

\underline{Fact}: There is no rational number $m$ such that $m^2 = 2$.

\hrulefill

\underline{Theorem}: There exists $\alpha \in \RR$ such that $\alpha^2 = 2$.

\hrulefill

\underline{Fact}: $\QQ$ is countable.

\hrulefill

\underline{Theorem}: $\RR$ is uncountable.

\pagebreak

\textbf{Chapter 2: Sequences - Definitions}

\hrulefill

A \defword{sequence} is a function whose domain is $\NN$ ($f : \NN \rightarrow \RR$).

\hrulefill

A sequence $(a_n)$ \defword{converges} to a real number $a$ (denoted $\lim a_n = a$) if

$\forall_{\eps > 0} \exists_{n_0 \in \NN} \forall_{n \ge n_0} \text{ } \abs{a_n - a} < \eps$.

\hrulefill

If a sequence does not converge for any real number, then the sequence \defword{diverges}.

\hrulefill

A sequence $(a_n)$ is \defword{bounded} if there exists $M > 0$ such that $\abs{a_n} \le M$ for every $n \in \NN$.

\hrulefill

A sequence $(a_n)$ is \defword{increasing} if $a_n \le a_{n + 1}$ for every $n \in \NN$.

A sequence $(a_n)$ is \defword{decreasing} if $a_n \ge a_{n + 1}$ for every $n \in \NN$.

\hrulefill

A sequence $(a_n)$ is \defword{bounded from above} if there exists $M \in \RR$ such that $a_n \le M$ for every $n \in \NN$.

A sequence $(a_n)$ is \defword{bounded from below} if there exists $M \in \RR$ such that $a_n \ge M$ for every $n \in \NN$.

\hrulefill

A sequence is \defword{monotone} if it is either increasing or decreasing.

\hrulefill

Let $(a_n)$ be a sequence of real numbers and let $n_1 < n_2 < \dots$ be an increasing (but not necessarily consecutive) sequence of natural numbers. Then
\\
$(a_{n_k}) = (a_{n_1}, a_{n_2}, \dots)$ is called a \defword{subsequence} of $(a_n)$.

\hrulefill

A sequence $(a_n)$ is called a \underline{Cauchy sequence} if for every $\eps > 0$ there is $n_0 \in \NN$ such that for all $m, n \ge n_0, \abs{a_n - a_m} < \eps$.

\pagebreak

Let $(b_n)$ be a sequence. An \defword{infinite series} is the expression $\Sigma_{n = 1}^{\infty} b_n = b_1 + b_2 + \cdots$.

The sequence of \defword{partial sums} $(s_m)$ is such that $s_m = b_1 + \cdots + b_m$.

We say that $\Sigma_{n = 1}^{\infty} b_n$ \defword{converges} to $B$ if $(s_m)$ converges to $B$.
\\
Then we write $\Sigma_{n = 1}^{\infty} b_n = B$.

\pagebreak

\textbf{Chapter 2: Sequences - Facts}

\hrulefill

\underline{Theorem}: If a limit exists, then it is unique.

\hrulefill

\underline{Theorem}: Every convergent sequence is bounded.

\hrulefill

\underline{Theorem}: Algebraic Limit Theorem

Let $\lim a_n = a$ and $\lim b_n = b$. Then:

\begin{enumerate}
\item $\lim (c \cdot a_n) = ca$, for every $c \in \RR$
\item $\lim (a_n + b_n) = a + b$
\item $\lim (a_n b_n) = ab$
\item $\lim \parens{\frac{a_n}{b_n}} = \frac{a}{b}$, provided $b \ne 0$
\end{enumerate}

\hrulefill

\underline{Theorem}: Order Limit Theorem

Let $\lim a_n = a$ and $\lim b_n = b$. Then:

\begin{enumerate}
\item If $a_n \ge 0$ for every $n$, then $a \ge 0$.
\item If $a_n \le b_n$ for every $n$, then $a \le b$.
\item If there exists $c$ such that $c \le b_n$, for all $n$, then $c \le b$.
\\
Similarly, if there exists $c$ such that $a_n \le c$, then $a \le c$.
\end{enumerate}

\hrulefill

\underline{Theorem}: Monotone Convergence Theorem

If a sequence is monotone and bounded, then it converges.

\hrulefill

$(\ast)$ \underline{Theorem}: MCT for increasing sequences

Let $(a_n)$ be an increasing sequence that is bounded from above. Then $(a_n)$ converges.

\pagebreak

\underline{Theorem}: Subsequences of a convergent sequence converge to the same limit as the original sequence.

\hrulefill

$(\ast)$ \underline{Theorem}: Bolzano-Weierstrass Theorem

Every bounded sequence contains a convergent subsequence.

\hrulefill

\underline{Lemma}: Every sequence contains a monotone subsequence.

\hrulefill

\underline{Theorem}: Every convergent sequence is a Cauchy sequence.

\hrulefill

\underline{Lemma}: Every Cauchy sequence is bounded.

\hrulefill

$(\ast)$ \underline{Theorem}: Cauchy Criterion

A sequence converges if and only if it is a Cauchy sequence.

\pagebreak

\textbf{Chapter 4: Continuity - Definitions}

\hrulefill

Let $x \in \RR$ and let $\eps > 0$. Then the \defword{$\eps$-neighborhood} is defined as:
\\
$V_{\eps}(x) = \set{y \mid \abs{x - y} < \eps}$.

\hrulefill

Let $A \subseteq \RR$. Then $x$ is called a \defword{limit point} of $A$ if for every $\eps > 0$, there exists $z \in A \cap V_{\eps}(x)$ such that $z \ne x$.

\hrulefill

Let $A$ be a non-degenerate interval or an interval with one point removed. Then the \defword{limit points of $A$} are points of $A$ together with the endpoints (if finite) and the removed point.

\hrulefill

Let $f : A \rightarrow \RR$ and let $c$ be a limit point of $A$. Then the \defword{functional limit} $lim_{x \rightarrow c} f(x) = L$ if for every $\eps > 0$, there is $\delta > 0$ such that if $0 < \abs{x - c} < \delta$ and $x \in A$, then $\abs{f(x) - L} < \eps$.

$\lim_{x \rightarrow c} f(x) = L \equiv \forall_{\eps > 0} \exists_{\delta > 0} \forall_{x \in A} 0 < \abs{x - c} < \delta \rightarrow \abs{f(x) - L} < \eps$

\hrulefill

A function $f : A \rightarrow \RR$ is \defword{continuous at a point} $c \in A$ if for every $\eps > 0$, there exists $\delta > 0$ such that if $\abs{x - c} < \delta$ and $x \in A$, then $\abs{f(x) - f(c)} < \eps$.

$\forall_{\eps > 0} \exists_{\delta > 0} \forall_{x \in A} \abs{x - c} < \delta \rightarrow \abs{f(x) - f(c)} < \eps$

\hrulefill

$f : A \rightarrow \RR$ is \defword{continuous on $A$} if it is continuous at $c$ for every $c \in A$.

\hrulefill

A function $f : A \rightarrow \RR$ is \defword{uniformly continuous} on $A$ if for every $\eps > 0$, there exists $\delta > 0$ such that for every $x, y \in A$, if $\abs{x - y} < \delta$, then $\abs{f(x) - f(y)} < \eps$.

$\forall_{\eps > 0} \exists_{\delta > 0} \forall_{x, y \in A} \abs{x - y} < \delta \rightarrow \abs{f(x) - f(y)} < \eps$

\pagebreak

\textbf{Chapter 4: Continuity - Facts}

\hrulefill

\underline{Theorem}: Let $A \subseteq \RR$. Then $x$ is a limit point of $A$ if and only if there exists a sequence $(a_n)$ that is contained in $A$ and $\lim a_n = x$ and $a_n \ne x$ (for every $n \in \NN$).

\hrulefill

\underline{Theorem}: Let $f : A \rightarrow \RR$ and let $c$ be a limit point of $A$. The following statements are equivalent:

\begin{enumerate}
\item $\lim_{x \rightarrow c} f(x) = L$

\item For every sequence $(x_n) \subseteq A$ satisfying $x_n \ne c$ and $\lim x_n = c$, we have $\lim f(x_n) = L$.
\end{enumerate}

\hrulefill

\underline{Corollary}: Let $f, g : A \rightarrow \RR$ and suppose $\lim_{x \rightarrow c} f(x) = L, \lim_{x \rightarrow c} g(x) = M$. Then:

\begin{enumerate}
\item $\lim_{x \rightarrow c} k \cdot f(x) = k \cdot L$ for every $k \in \RR$

\item $\lim_{x \rightarrow c} (f(x) + g(x)) = L + M$

\item $\lim_{x \rightarrow c} (f(x)g(x)) = L \cdot M$

\item $\lim_{x \rightarrow c} \frac{f(x)}{g(x)} = \frac{L}{M}$ provided $M \ne 0$
\end{enumerate}

\hrulefill

\underline{Corollary}: Divergence Criterion for Functional Limits

Let $f : A \rightarrow \RR$ and let $c$ be a limit point of $A$. If there exist two sequences $(x_n), (y_n)$ in $A$ such that $x_n \ne c, y_n \ne c, \lim x_n = \lim y_n = c$ but $\lim f(x_n) \ne \lim f(y_n)$, then $\lim_{x \rightarrow c} f(x)$ does not exist.

\pagebreak

\underline{Theorem}: Characterization of Continuity

Let $f : A \rightarrow \RR$ and let $c \in A$. The function is continuous at $c$ if and only if any of the following conditions is satisfied:

\begin{enumerate}
\item $\forall_{\eps > 0} \exists_{\delta > 0} (\abs{x - c} < \delta \land x \in A) \rightarrow \abs{f(x) - f(c)} < \eps$

\item For all $V_\eps(f(c))$, there exists $V_\delta(c)$ such that $x \in V_\delta(c) \cap A$ implies $f(x) \in V_{\eps}(f(c))$.

\item For all $(x_n) \rightarrow c$ with $x_n \in A, (f(x_n)) \rightarrow f(c)$.

In addition, if $c$ is a limit point of $A$, then the above are equivalent to:

\item $\lim_{x \rightarrow c} f(x) = f(c)$
\end{enumerate}

\hrulefill

\underline{Corollary}: Criterion for Discontinuity

Let $f : A \rightarrow \RR$ and let $c \in A$ (be a limit point of $A$). If there is a sequence $(x_n) \subseteq A$ such that $(x_n) \rightarrow c$ but $(f(x_n))$ does not converge to $f(c)$, then $f$ is not continuous at $c$.

\hrulefill

\underline{Theorem}: Algebraic Continuity Theorem

Let $f, g : A \rightarrow \RR$ be continuous at $c \in A$. Then:

\begin{enumerate}
\item $k \cdot f(x)$ is continuous at $c$ for every $k \in \RR$.

\item $f(x) + g(x)$ is continuous at $c$.

\item $f(x)g(x)$ is continuous at $c$.

\item $\frac{f(x)}{g(x)}$ is continuous at $c$ provided that the quotient is defined.
\end{enumerate}

\hrulefill

\underline{Theorem}: Let $f : A \rightarrow \RR$ and $g : B \rightarrow \RR$ be such that $f(A) \subseteq B$. If $f, g$ are continuous, then $g \circ f$ is a continuous function from $A$ to $\RR$.

\pagebreak

\underline{Theorem}: Bolzano-Weierstrass, version 2

Let $A$ be a closed bounded interval. Every sequence $(x_n) \subseteq A$ contains a convergent subsequence whose limit is in $A$.

$A = [a, b], z = \lim x_{n_k}, a \le z \le b$

\hrulefill

$(\ast)$ \underline{Theorem}: Extreme Value Theorem

Let $f : [a, b] \rightarrow \RR$ be continuous. Then there exist $\alpha, \beta \in [a, b]$ such that for all $x \in [a, b], f(\alpha) \le f(x) \le f(\beta)$.

\hrulefill

\underline{Theorem}: A function $f : A \rightarrow \RR$ fails to be uniformly continuous on $A$ if and only if there exists $\eps_0 > 0$ and two sequences $(x_n), (y_n)$ in $A$ satisfying $\abs{x_n - y_n} \rightarrow 0$ but $\abs{f(x_n) - f(y_n)} \ge \eps_0$.

\hrulefill

$(\ast)$ \underline {Theorem}: Let $f : [a, b] \rightarrow \RR$ be continuous. Then $f$ is uniformly continuous on $[a, b]$.

\hrulefill

$(\ast)$ \underline{Theorem}: Bolzano's Theorem

Let $f : [a, b] \rightarrow \RR$ be continuous. Suppose $f(a) < 0$ and $f(b) > 0$. Then there is $c \in (a, b)$ such that $f(c) = 0$.

\hrulefill

\underline{Theorem}: Intermediate Value Theorem

Let $f : [a, b] \rightarrow \RR$ be continuous. Let $L \in \RR$ be such that $\min \set{f(a), f(b)} < L < \max \set{f(a), f(b)}$. Then there is $c \in (a, b)$ such that $f(c) = L$.

\hrulefill

\underline{Theorem}: Let $f : [a, b] \rightarrow \RR$ be continuous. Then $f([a, b])$ (the range of $f$) is a closed interval.

\hrulefill

\underline{Continuity of the Inverse}: Let $f : [a, b] \rightarrow \RR$ be continuous and injective. Then $f([a, b]) = [c, d]$ and $f^{-1} : [c, d] \rightarrow [a, b]$.

\underline{Theorem}: With the setup as above, $f^{-1} : [c, d] \rightarrow [a, b]$ is continuous.

\pagebreak

\textbf{Chapter 5: The Derivative - Definitions}

\hrulefill

Let $g : A \rightarrow \RR$ be a function defined on interval $A$ and let $c \in A$. The \defword{derivative} of $g$ at $c$ is (provided that the limit exists):
\[
g'(c) = \lim \limits_{x \rightarrow c} \frac{g(x) - g(c)}{x - c}
\]

If $g$ exists for all points $c \in A$, then $g$ is \defword{differentiable} on $A$.

\pagebreak

\textbf{Chapter 5: The Derivative - Facts}

\hrulefill

\underline{Theorem}: If $g : A \rightarrow \RR$ is differentiable at $c$, then $g$ is continuous at $c$.

\hrulefill

\underline{Theorem}: Let $f, g : A \rightarrow \RR$ are differentiable at $c \in A$. Then:

\begin{enumerate}
\item $(f + g)'(c) = f'(c) + g'(c)$

\item For $k \in \RR, (k \cdot f)'(c) = k \cdot f'(c)$

\item $(fg)'(c) = f'(c)g(c) + f(c)g'(c)$ (Product Rule)

\item $\parens{\frac{f}{g}}'(c) = \frac{f'(c)g(c) - f(c)g'(c)}{(g(c))^2}$ provided $g(c) \ne 0$ (Quotient Rule)
\end{enumerate}

\hrulefill

\underline{Theorem}: The Chain Rule

Let $f : A \rightarrow B, g : B \rightarrow \RR$ satisfy $f(A) \subseteq B$. If $f$ is differentiable at $c \in A$ and $g$ is differentiable at $f(c) \subseteq B$, then $g \circ f$ is differentiable at $c$ and
\[
(g \circ f)'(c) = g'(f(c)) f'(c)
\]

\hrulefill

\underline{Theorem}: Interior Extremum Theorem

Let $f$ be differentiable on $(a, b)$.

\begin{enumerate}
\item If $f$ attains a maximum value at some point $c \in (a, b)$, then $f'(c) = 0$.

\item If $f$ attains a minimum value at some point $c \in (a, b)$, then $f'(c) = 0$.
\end{enumerate}

\end{document}
