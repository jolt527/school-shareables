%% Thanks to Bennet Goeckner for giving me his TeX template, which this is based on. 
%% These percent symbols tell the compiler to ignore the remainder of a given line.
%% We use them to write comments that will not appear in the finalized output.

%% The following tells the compiler which type of document we're making.
%% There are many options. ``Article'' is probably fine for our class.
\documentclass[12pt]{article}

%% After declaring the documentclass, we load some packages which give us 
%% some built-in commands and more functionality. 
%% The following is a list of packages that this file might use.
%% If a command you're using isn't working, try Googling it -- you might need to add a specific package.
%% I have included the standard ones that I like to load.
\usepackage[table]{xcolor}
\usepackage{amsmath}
\usepackage{amsthm}
\usepackage{amsfonts}
\usepackage{amssymb}
\usepackage{enumerate}
\usepackage{graphicx}
\usepackage{mdframed}
\usepackage{multicol}
\usepackage{verbatim}
\usepackage{tikz}
\usepackage[margin = .8in]{geometry}
\geometry{letterpaper}
\linespread{1.2}
\usepackage{cancel}

%% One of the nicest things about LaTeX is you can create custom macros. If  there is a long-ish expression that you will write often, it is nice to give it a shorter command.
%% For our common number systems.
\newcommand{\RR}{\mathbb{R}} %% The blackboard-bold R that you have seen used for real numbers is typeset by $\mathbb{R}$. This macro means that $\RR$ will yield the same result, and is much shorter to type.
\newcommand{\NN}{\mathbb{N}}
\newcommand{\ZZ}{\mathbb{Z}} 
\newcommand{\QQ}{\mathbb{Q}}

%% Your macros can even accept arguments. 
\newcommand\set[1]{\left\lbrace #1 \right\rbrace} %% In mathmode, if you write \set{STUFF}, then this will output {STUFF}, i.e. STUFF inside of a set
\newcommand\abs[1]{\left| #1 \right|} %% This will do the same but with vertical bars. I.e., \abs{STUFF} gives |STUFF|
\newcommand\parens[1]{\left( #1 \right)} %% Similar. \parens{STUFF} gives (STUFF)
\newcommand\brac[1]{\left[ #1 \right]} %% Similar. \brac{STUFF} gives [STUFF]
\newcommand\sol[1]{\begin{mdframed}
\emph{Solution.} #1
\end{mdframed}}
\newcommand\solproof[1]{\begin{mdframed}
\begin{proof} #1
\end{proof}
\end{mdframed}}

\newcommand\augmat[2]{\left[ \begin{array}{#1} #2 \end{array} \right]}
\newcommand\matopsinterchange[2]{#1 \longleftrightarrow #2}
\newcommand\matopsreplace[2]{#1 \xrightarrow{} #2}
\newcommand\matopsarrow[1]{\xrightarrow[\substack{#1}]{}}

\renewcommand\qedsymbol{$\blacksquare$}

\newcommand{\Pc}{\mathcal{P}}

\newcommand{\eps}{\varepsilon}

\newcommand\ceil[1]{\left\lceil #1 \right\rceil}
\newcommand\floor[1]{\left\lfloor #1 \right\rfloor}

\graphicspath{ {./images/} }

%% A few more important commands:

%% You should start every proof with \begin{proof} and end it with \end{proof}.  
%%
%% Code inside single dollar signs will give in-line mathmode. I.e., $f(x) = x^2$ 
%% Code \[ \] will give mathmode centered on its own line.
%%
%% Other common commands:
%%	\begin{align*} and \end{align*} -- Good for multiline equations
%%	\begin{align} and \end{align} -- Same as above, but it will number the equations for easy reference
%%	\emph{italicized text here} and \textbf{bold text here} are also useful.
%%
%% Some very specific mathmode commands and their meanings:
%%	x \in A -- x is an element of A
%%	x \notin A -- x is not an element of A
%%	A \subseteq B -- A is a subset of B
%%	A \subsetneq B -- A is a proper subset of B
%%	x \equiv y \pmod{n} -- x is congruent to y mod n. 
%%	x \geq y and x \leq y -- Greater than or equal to and less than or equal to 
%%
%% You'll probably find lots of relevant commands in the question prompts. Also Google is your friend!

\begin{document}

\large
\textbf{2.2 The Limit of a Sequence}

\hrulefill

\begin{enumerate}

% 2.2.1
\item
\begin{enumerate}
\item
What happens if we reverse the order of the quantifiers in Definition 2.2.3?

\textit{Definition:} A sequence $(x_n)$ \textit{verconges} to $x$ if \textit{there exists} an $\eps > 0$ such that \textit{for all} $N \in \NN$ is it true that $n \ge N$ implies $\abs{x_n - x} < \eps$.

\begin{enumerate}
\item Give an example of a vercongent sequence.

$x_n = \frac{1}{n}$ verconges to 1.

\begin{proof}
$ $

Let $\eps = 2$.

Let $N \in \NN$ be arbitrary.

For every $n \ge N, \abs{\frac{1}{n} - 1} = \abs{1 - \frac{1}{n}} = 1 - \frac{1}{n} \le 1 < 2 = \eps$.
\end{proof}

\item Is there an example of a vercongent sequence that is divergent?

The divergent sequence $x_n = (-1)^n$ verconges to 1.

\begin{proof}
$ $

Let $\eps = 3$.

Let $N \in \NN$ be arbitrary.

For every $n \ge N, \abs{(-1)^n - 1} \le 2 < 3 = \eps$.
\end{proof}

\item Can a sequence verconge to two different values?

Yes. For example, $x_n = \frac{1}{n}$ verconges to 1 (already shown above) and also 2.

\begin{proof}
$ $

Let $\eps = 2$.

Let $N \in \NN$ be arbitrary.

For every $n \ge N, \abs{\frac{1}{n} - 2} = \abs{2 - \frac{1}{n}} = 2 - \frac{1}{n} < 2 = \eps$.
\end{proof}

\item What exactly is being described in this strange definition?

A vercongence sequence is a bounded sequence.
\end{enumerate}
\end{enumerate}

\hrulefill

% 2.2.2
\item
Verify, using the definition of converge of a sequence, that the following sequences converge to the proposed limit.
\begin{enumerate}
\item $\lim \frac{2n + 1}{5n + 4} = \frac{2}{5}$.

\begin{proof}
$ $

Let $\eps > 0$ be arbitrary.

By the ArP, there exists $n_0 \in \NN$ such that $\frac{1}{n_0} < \eps$.

For every $n \ge n_0, \abs{\frac{2n + 1}{5n + 4} - \frac{2}{5}} = \abs{\frac{10n + 5 - 10n - 8}{5(5n + 4)}} = \abs{\frac{-3}{5(5n + 4)}} = \frac{3}{5} \cdot \frac{1}{5n + 4}
\\
< \frac{3}{5} \cdot \frac{1}{5n} = \frac{3}{25} \cdot \frac{1}{n} < \frac{1}{n} \le \frac{1}{n_0} < \eps$.
\end{proof}

\item $\lim \frac{2 n^2}{n^3 + 3} = 0$.

\begin{proof}
$ $

Let $\eps > 0$ be arbitrary.

By the ArP, there exists $n_0 \in \NN$ such that $\frac{1}{n_0} < \frac{\eps}{2}$, and so $\frac{2}{n_0} < \eps$.

For every $n \ge n_0, \abs{\frac{2 n^2}{n^3 + 3} - 0} = \abs{\frac{2 n^2}{n^3 + 3}} = \frac{2 n^2}{n^3 + 3} < \frac{2 n^2}{n^3} = \frac{2}{n} \le \frac{2}{n_0} < \eps$.
\end{proof}

\item $\lim \frac{\sin (n^2)}{\sqrt[3]{n}} = 0$.

\begin{proof}
$ $

Let $\eps > 0$ be arbitrary.

By the ArP, there exists $n_0 \in \NN$ such that $\frac{1}{n_0} < \eps$.

For every $n \ge n_0, \abs{\frac{\sin (n^2)}{\sqrt[3]{n}} - 0} = \abs{\frac{\sin (n^2)}{\sqrt[3]{n}}} \le \frac{1}{\sqrt[3]{n}} \le \frac{1}{n} \le \frac{1}{n_0} < \eps$.
\end{proof}
\end{enumerate}

% 2.2.3
\item Describe what we would have to demonstrate in order to disprove each of the following statements.
\begin{enumerate}
\item At every college in the United States, there is a student who is at least seven feet tall.

Find a college in the United States in which every student is less than seven feet tall.

\item For all colleges in the United States, there exists a professor who gives every student a grade of either A or B.

Find a college in the United States in which every professor gives a student a grade that is not an A and is not a B.

\item There exists a college in the United States where every student is at least six feet tall.

Show that every college in the United States has a student that is less than six feet tall.
\end{enumerate}

% 2.2.4
\item Give an example of each or state that the request is impossible. For any that are impossible, give a compelling argument for why that is the case.
\begin{enumerate}
\item A sequence with an infinite number of ones that does not converge to one.

$(x_n) = ((-1)^n) = (-1, 1, -1, 1, -1, 1, \dots)$ has an infinite number of ones, but diverges.

\item A sequence with an infinite number of ones that converges to a limit not equal to one.

This is impossible.

\begin{proof}
$ $

By contradiction.

Suppose $(x_n)$ is a sequence with an infinite number of ones that converges to a limit not equal to 1.

Let $(x_{n_k})$ be a subsequence of $(x_n)$ such that $n_k$ is the $k^{\text{th}}$ one in $(x_n)$.

Since $x_{n_k} = 1$ for every $k \in \NN, \lim x_{n_k} = 1$.

Thus, $(x_n)$ also converges to 1, but this contradicts the assumption that its limit is not equal to 1.
\end{proof}


\item A divergent sequence such that for every $n \in \NN$ it is possible to find $n$ consecutive ones somewhere in the sequence.

$(x_n) = (0, 1, 0, 1, 1, 0, 1, 1, 1, 0, 1, 1, 1, 1, 0, \dots)$
\end{enumerate}

% 2.2.5
\item Let $[[x]]$ be the greatest integer less than or equal to $x$. For example, $[[\pi]] = 3$ and $[[3]] = 3$. For each sequence, find $\lim a_n$ and verify it with the definition of convergence.
\begin{enumerate}
\item $a_n = [[5/n]]$

$\lim a_n = 0$

\begin{proof}
$ $

Let $\eps > 0$ be arbitrary.

Let $n_0 = 6$.

For $n \ge n_0, \abs{\brac{\brac{\frac{5}{n}}} - 0} = \abs{\brac{\brac{\frac{5}{n}}}} = \brac{\brac{\frac{5}{n}}} = 0 < \eps$.
\end{proof}

\item $a_n = [[(12 + 4n) / 3n]]$

$\lim a_n = 1$

\begin{proof}
$ $

Let $\eps > 0$ be arbitrary.

Let $n_0 = 7$.

For $n \ge n_0, \abs{\brac{\brac{\frac{12 + 4n}{3n}}} - 1} = \abs{\brac{\brac{\frac{12 + 4n}{3n} - 1}}} = \abs{\brac{\brac{\frac{12 + 4n - 3n}{3n}}}}
\\
= \abs{\brac{\brac{\frac{12 + n}{3n}}}} = \brac{\brac{\frac{12 + n}{3n}}} = 0 < \eps$.
\end{proof}
\end{enumerate}

Reflecting on these examples, comment on the statement following Definition 2.2.3 that ``the smaller the $\eps$-neighborhood, the larger $N$ may have to be.''

These examples show that after a certain $n_0$ value (or $N$ value in the words of the statement), the values of the sequence are equal to the limit value, regardless of the $\eps$-value.

% 2.2.6
\item Prove Theorem 2.2.7. To get started, assume $(a_n) \rightarrow a$ and also that
\\
$(a_n) \rightarrow b$. Now argue $a = b$.

\begin{proof}
$ $

Let $\eps > 0$ be arbitrary.

Suppose $a, b \in \RR$ and $\lim a_n = a$ and $\lim a_n = b$.

Then there exists $n_1 \in \NN$ such that for all $n \ge n_1, \abs{a_n - a} < \frac{\eps}{2}$.

Then there exists $n_2 \in \NN$ such that for all $n \ge n_2, \abs{a_n - b} < \frac{\eps}{2}$.

Let $n_0 = \max \set{n_1, n_2}$.

Then for all $n \ge n_0, \abs{a - b} = \abs{a - a_n + a_n - b} \le \abs{a_n - a} + \abs{a_n - b}
\\
< \frac{\eps}{2} + \frac{\eps}{2} = \eps$.

Thus, $a - b = 0$, and so $a = b$.
\end{proof}

% 2.2.7
\item Here are two useful definitions:
\begin{enumerate}
\item A sequence $(a_n)$ is \textit{eventually} in a set $A \subseteq \RR$ if there exists an $N \in \NN$ such that $a_n \in A$ for all $n \ge N$.

\item A sequence $(a_n)$ is \textit{frequently} in a set $A \subseteq \RR$ if, for every $N \in \NN$, there exists an $n \ge N$ such that $a_n \in A$.

\begin{enumerate}
\item Is the sequence $(-1)^n$ eventually or frequently in the set $\set{1}$?

\begin{enumerate}
\item
Frequently

\begin{proof}
$ $

Let $N \in \NN$ be arbitrary.

If $N$ is even, then let $n = N$, and so $n$ is even.

If $N$ is odd, then let $n = N + 1$, and so $n$ is even.

Then $a_n = (-1)^n = 1 \in \set{1}$.
\end{proof}

\item
Not eventually

\begin{proof}
$ $

By contradiction.

Suppose $a_n$ is eventually in $A$.

Then there exists $N \in \NN$ such that for every $n \ge N, a_n \in A$.

If $N$ is even, then let $m = N + 1$, and so $m$ is odd.

If $N$ is odd, then let $m = N + 2$, and so $m$ is odd.

Since $m \ge N$, then it must be that $m \in A$.

However, $a_m = (-1)^m = -1 \notin A$, which is a contradiction.
\end{proof}
\end{enumerate}

\item Which definition is stronger? Does frequently imply eventually or does eventually imply frequently?

Eventually is stronger, and eventually implies frequently:

\begin{proof}
$ $

Suppose $(a_n)$ is a sequence that is eventually in a set $A \subseteq \RR$.

Then there exists $n_0 \in \NN$ such that for every $n \ge n_0, a_n \in A$.

Let $N \in \NN$ be arbitrary.

If $N \ge n_0$, let $n = N$, and so $n \ge N$.

If $N < n_0$, let $n = n_0$, and so $n \ge N$.

Thus, $a_n \in A$.

Therefore, $(a_n)$ is a sequence that is frequently in a set $A \subseteq \RR$.
\end{proof}

\item Give an alternate rephrasing of Definition 2.2.3B using either frequently or eventually. Which is the term we want?

\textbf{Definition 2.2.3B.} A sequence $(a_n)$ converges to $a$ if, given any $\eps$-neighborhood $V_{\eps} (a)$ of $a$, there exists a point in the sequence after which all the terms are in $V_{\eps} (a)$.

\textbf{Alternate rephrasing.} A sequence $(a_n)$ converges to $a$ if, given any $\eps$-neighborhood $V_{\eps} (a)$ of $a$, $(a_n)$ is eventually in $V_{\eps} (a)$.

\item Suppose an infinite number of terms of a sequence $(x_n)$ are equal to 2.

\begin{enumerate}
\item
Is $(x_n)$ necessarily eventually in the interval $(1.9, 2.1)$?

Not necessarily, as $x_n = (-1)^n + 1$ has terms that are oscillating between 0 and 2, and so there are an infinite number of 2s in $(x_n)$, but there is no $N \in \NN$ at and after which there are no 0s in $(x_n)$.

\item
Is it frequently in $(1.9, 2.1)$?

Yes, for every $N \in \NN$, we can find an $n \ge N$ such that
\\
$a_n = 2 \in (1.9, 2.1) = A$.
\end{enumerate}

\end{enumerate}
\end{enumerate}

% 2.2.8
\item For some additional practice with nested quantifiers, consider the following invented definition:

Let's call a sequence $(x_n)$ \textit{zero-heavy} if there exists $M \in \NN$ such that for all $N \in \NN$ there exists $n$ satisfying $N \le n \le N + M$ where $x_n = 0$.
\begin{enumerate}
\item Is the sequence $\set{0, 1, 0, 1, 0, 1, \dots}$ zero heavy?

Yes.

\begin{proof}
$ $

Let $M = 1$.

Let $N \in \NN$ be arbitrary.

If $N$ is odd, then let $n = N$, and so $N \le n \le N + 1 = N + M$ and $n$ is odd and $x_n = 0$.

If $N$ is even, then let $n = N + 1$, and so $N \le n \le N + 1 = N + M$ and $n$ is odd and $x_n = 0$.

Therefore, $\set{0, 1, 0, 1, 0, 1, \dots}$ is zero heavy.
\end{proof}

\item If a sequence is zero-heavy does it necessarily contain an infinite number of zeros? If not, provide a counterexample.

Yes.

\item If a sequence contains an infinite number of zeros, is it necessarily zero-heavy? If not, provide a counterexample.

No. For example:

\[
x_n =
\begin{cases}
	0 & \text{if } n \text{ is a perfect square} \\
	1 & \text{otherwise} \\
\end{cases}
\]

\begin{proof}
$ $

Let $M \in \NN$ be arbitrary.

Let $k \in \NN$ be such that $k > M$.

Let $N = k^2 + 1$.

Then $k^2 + 1 \le N + M = k^2 + 1 + M < k^2 + k + 1 < k^2 + 2k + 1 < (k + 1)^2$.

Then for every $n$ satisfying $N \le n \le N + M, x_n = 1$.

Therefore, $(x_n)$ is not zero-heavy.
\end{proof}

\item Form the logical negation of the above definition. That is, complete the sentence: A sequence is \textit{not} zero-heavy if \dots.

A sequence is not zero-heavy if for every $M \in \NN$, there exists $N \in \NN$ such that for every $n$ satisfying $N \le n \le N + M, x_n \ne 0$.
\end{enumerate}

\end{enumerate}

\end{document}
